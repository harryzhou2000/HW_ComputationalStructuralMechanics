%!TEX program = xelatex
\documentclass[UTF8,c5size]{ctexart}


\title{{\bfseries 第三次作业}}
\author{周涵宇 2018011600}
\date{}

\usepackage[a4paper]{geometry}
\geometry{left=0.75in,right=0.75in,top=1in,bottom=1in}

\usepackage[
UseMSWordMultipleLineSpacing,
MSWordLineSpacingMultiple=1.5
]{zhlineskip}

\usepackage{fontspec}
\setmainfont{Cambria Math}
% \setmonofont{JetBrains Mono}
\setCJKmainfont{仿宋}[AutoFakeBold=true]
\setCJKsansfont{黑体}[AutoFakeBold=true]

\usepackage{bm}
\usepackage{amsmath}
\usepackage{array}

\begin{document}

\maketitle

考虑铁木辛哥梁直梁单元,$x$ 为梁的方向。由于只考虑了小变形和直梁,
弯曲可以单独求解,以下只考虑一个平面内的弯曲自由度,即考虑自由度
$w$、$\theta_y=-\frac{dw}{dx}$。

% \begin{equation}
% \bm{a_e}=[u^1\ \ v^1\ \ w^1\ \ \theta^1_x\ \ \theta^1_y\ \ \theta^1_z
% \ \ u^2\ \ v^2\ \ w^2\ \ \theta^2_x\ \ \theta^2_y\ \ \theta^2_z]^T
% \end{equation}

拉格朗日2节点一维插值为:

\begin{equation}
    N^1(\xi) = \frac{1-\xi}{2},\ \ N^2(\xi) = \frac{1+\xi}{2}
\end{equation}

拉格朗日3节点一维插值为:

\begin{equation}
    N^1(\xi) = \frac{\xi(\xi-1)}{2},\ \ 
    N^2(\xi) = 1-\xi^2,\ \ 
    N^3(\xi) = \frac{\xi(\xi+1)}{2}
\end{equation}


以上插值都定义在$[-1,1]$。

铁木辛哥直梁内关于面内弯曲的势能为:

\begin{equation}
    \begin{split}
        \Pi_p=
            &\int_{l}{\left[
            \frac{1}{2}EI_y\left(\frac{d\theta_y}{dx}\right)^2+  
            \frac{1}{2}\frac{GA}{k}\left(\frac{dw}{dx}+\theta_y\right)^2
            \right]dl}+\\
            &-\int_{l}{\left[
                wP_z+\theta_yM_y
                \right]dl}\\
            &-\sum_{\forall i_c}{\left[
            w_{i_c}P_{zi_c}
            +\theta_{yi_c}M_{yi_c}
            \right]}
    \end{split}
\end{equation}

注意此处的力矩载荷都是定义为y轴右手方向,与$\frac{dw}{dx}$不同。

势能变分原理为

$$
\delta\Pi_p=0
$$

其中要求满足位移可能。

变分并取形函数近似则有:

\begin{equation}
    (\bm{K_b}+\bm{K_s})\bm{a}=\bm{P}    
\end{equation}

拆分为单元则:

\begin{equation}
    \begin{split}
        \bm{K_b^e}&=\int_{l}{EI_y\left(\frac{d\bm{N_\theta}}{dx}\right)^T\left(\frac{d\bm{N_\theta}}{dx}\right)dl}\\
        \bm{K_s^e}&=\int_{l}{\frac{GA}{k}\bm{B_s}^T\bm{B_s}dl}\\
        \bm{P^e}&=\int_{l}{\left[
            \bm{N_w}^TP_z+\bm{N_\theta}^TM_y
            \right]dl}+
            \sum_{i_c}{\left[
                \bm{N_w(\xi_{ic})}^TP_{zi_c}+\bm{N_\theta(\xi_{ic})}^TM_{yi_c}
            \right]}
    \end{split}
\end{equation}

其中给出了形函数矩阵
以及局部自由度排列为:

\begin{equation}
    \begin{split}
        \bm{N_w}&=
        \begin{bmatrix}
            N^1&0&N^2&0&\dotsb
        \end{bmatrix}\\
        \bm{N_\theta}&=
        \begin{bmatrix}
            0&N^1&0&N^2&\dotsb
        \end{bmatrix}\\
        \bm{B_s}&=\frac{d\bm{N_w}}{dx}+\bm{N_\theta}\\
        \bm{a^e}&=
        \begin{bmatrix}
            w^1&\theta^1&w^2&\theta^2&\dotsb
        \end{bmatrix}^T
    \end{split}
\end{equation}

应对剪切锁死的方案是采用减缩积分,因此给出几种一维的高斯积分方案:

\begin{equation}
    \begin{split}
        1).& \bm{\xi_p}=\begin{bmatrix}
            0
        \end{bmatrix}, \ 
        \bm{w_p}=\begin{bmatrix}
            2
        \end{bmatrix}
        \\ 2).& \bm{\xi_p}=\begin{bmatrix}
            -\frac{1}{\sqrt{3}}&\frac{1}{\sqrt{3}}
        \end{bmatrix}, \ 
        \bm{w_p}=\begin{bmatrix}
            1&1
        \end{bmatrix}
        \\ 3).& \bm{\xi_p}=\begin{bmatrix}
            -0.774596669241483&0&0.774596669241483
        \end{bmatrix}, \ 
        \bm{w_p}=\begin{bmatrix}
            \frac{5}{9}&\frac{8}{9}&\frac{5}{9}
        \end{bmatrix}
    \end{split}
\end{equation}

\begin{equation}
    \begin{split}
        A^H&=A\\
        AV&=VD\\
        v^HAv&=v^Hv\lambda\\
        \overline{v^HAv}&=v^TA^Tv^HT=(v^HAv)^T=v^HAv=\text{somereal}\\
        v^Hv&=\text{somereal}\\
        \Rightarrow \lambda&=\text{somereal}\\
        u^HAv&=u^Hv\lambda_v=v^HAu=v^Hu\lambda_u=\overline{u^Hv}\lambda_u\\
        \Rightarrow u^Hv&=0
    \end{split}
\end{equation}


\end{document}