%!TEX program = xelatex
\documentclass[UTF8,c5size]{ctexart}


\title{{\bfseries 第三次作业}}
\author{周涵宇 2018011600}
\date{}

\usepackage[a4paper]{geometry}
\geometry{left=0.75in,right=0.75in,top=1in,bottom=1in}

\usepackage[
UseMSWordMultipleLineSpacing,
MSWordLineSpacingMultiple=1.5
]{zhlineskip}

\usepackage{fontspec}
\setmainfont{Cambria Math}
% \setmonofont{JetBrains Mono}
\setCJKmainfont{仿宋}[AutoFakeBold=true]
\setCJKsansfont{黑体}[AutoFakeBold=true]

\usepackage{bm}
\usepackage{amsmath}
\usepackage{array}
\usepackage{multirow}
\usepackage{multicol}

\usepackage{ctex}

\begin{document}

\maketitle

\subsection*{1.铁木辛哥直梁的格式}
考虑铁木辛哥梁直梁单元,$x$ 为梁的方向。由于只考虑了小变形和直梁,
弯曲可以单独求解,以下只考虑一个平面内的弯曲自由度,即考虑自由度
$w$、$\theta_y=-\frac{dw}{dx}$。

% \begin{equation}
% \bm{a_e}=[u^1\ \ v^1\ \ w^1\ \ \theta^1_x\ \ \theta^1_y\ \ \theta^1_z
% \ \ u^2\ \ v^2\ \ w^2\ \ \theta^2_x\ \ \theta^2_y\ \ \theta^2_z]^T
% \end{equation}

拉格朗日2节点一维插值为:

\begin{equation}
    N^1(\xi) = \frac{1-\xi}{2},\ \ N^2(\xi) = \frac{1+\xi}{2}
\end{equation}

拉格朗日3节点一维插值为:

\begin{equation}
    N^1(\xi) = \frac{\xi(\xi-1)}{2},\ \ 
    N^2(\xi) = 1-\xi^2,\ \ 
    N^3(\xi) = \frac{\xi(\xi+1)}{2}
\end{equation}


以上插值都定义在$[-1,1]$。

铁木辛哥直梁内关于面内弯曲的势能为:

\begin{equation}
    \begin{split}
        \Pi_p=
            &\int_{l}{\left[
            \frac{1}{2}EI_y\left(\frac{d\theta_y}{dx}\right)^2+  
            \frac{1}{2}\frac{GA}{k}\left(\frac{dw}{dx}+\theta_y\right)^2
            \right]dl}+\\
            &-\int_{l}{\left[
                wP_z+\theta_yM_y
                \right]dl}\\
            &-\sum_{\forall i_c}{\left[
            w_{i_c}P_{zi_c}
            +\theta_{yi_c}M_{yi_c}
            \right]}
    \end{split}
\end{equation}

注意此处的力矩载荷都是定义为y轴右手方向,与$\frac{dw}{dx}$不同。

势能变分原理为

$$
\delta\Pi_p=0
$$

其中要求满足位移可能。

变分并取形函数近似则有:

\begin{equation}
    (\bm{K_b}+\bm{K_s})\bm{a}=\bm{P}    
\end{equation}

拆分为单元则:

\begin{equation}
    \begin{split}
        \bm{K_b^e}&=\int_{l}{EI_y\left(\frac{d\bm{N_\theta}}{dx}\right)^T\left(\frac{d\bm{N_\theta}}{dx}\right)dl}\\
        \bm{K_s^e}&=\int_{l}{\frac{GA}{k}\bm{B_s}^T\bm{B_s}dl}\\
        \bm{P^e}&=\int_{l}{\left[
            \bm{N_w}^TP_z+\bm{N_\theta}^TM_y
            \right]dl}+
            \sum_{i_c}{\left[
                \bm{N_w(\xi_{ic})}^TP_{zi_c}+\bm{N_\theta(\xi_{ic})}^TM_{yi_c}
            \right]}
    \end{split}
\end{equation}

其中给出了形函数矩阵
以及局部自由度排列为:

\begin{equation}
    \begin{split}
        \bm{N_w}&=
        \begin{bmatrix}
            N^1&0&N^2&0&\dotsb
        \end{bmatrix}\\
        \bm{N_\theta}&=
        \begin{bmatrix}
            0&N^1&0&N^2&\dotsb
        \end{bmatrix}\\
        \bm{B_s}&=\frac{d\bm{N_w}}{dx}+\bm{N_\theta}\\
        \bm{a^e}&=
        \begin{bmatrix}
            w^1&\theta^1&w^2&\theta^2&\dotsb
        \end{bmatrix}^T
    \end{split}
\end{equation}

\subsection*{2.积分方案}

应对剪切锁死的方案是采用减缩积分,因此给出几种一维的高斯积分方案:

\begin{equation}
    \begin{split}
        1) & \bm{\xi_p}=\begin{bmatrix}
            0
        \end{bmatrix}, \ 
        \bm{w_p}=\begin{bmatrix}
            2
        \end{bmatrix}
        \\ 2) & \bm{\xi_p}=\begin{bmatrix}
            -\frac{1}{\sqrt{3}}&\frac{1}{\sqrt{3}}
        \end{bmatrix}, \ 
        \bm{w_p}=\begin{bmatrix}
            1&1
        \end{bmatrix}
        \\ 3) & \bm{\xi_p}=\begin{bmatrix}
            -0.774596669241483&0&0.774596669241483
        \end{bmatrix}, \ 
        \bm{w_p}=\begin{bmatrix}
            \frac{5}{9}&\frac{8}{9}&\frac{5}{9}
        \end{bmatrix}
    \end{split}
\end{equation}

2节点插值为1次多项式,3节点为2次多项式,对应弯曲刚度分布为0次、2次,
对应剪切刚度分布为2次、4次。不考虑Jacobian非均匀。因此,精确积分为
2节点采用2积分点,3节点采用3积分点,减缩积分保障弯曲刚度阵非奇异则2节点
采用1积分点,3节点采用2积分点。

\subsection*{3.算例描述}

梁的截面都采用$0.1m\times hm$的矩形,剪应力不均匀系数
$k=\frac{6}{5}$,即高度不同。$E=1GPa$, $\nu = 0$,全长
$L=1m$
涉及模态时
$\rho=1e3kg/m^3$。边界条件为一端固支。

以上条件下,$1N/m$的均布力载荷下,欧拉-伯努利梁的端部挠度是
$\frac{1.5\times10^{-8}}{h^3}m^4$,$1N\cdot m$的端部
弯矩载荷下,欧拉-伯努利梁的端部挠度是$\frac{6\times10^{-8}}{h^3}m^4$。

对于均布载荷$P$的铁木辛哥梁,

\begin{equation}
    \begin{split}
        M_y&=-\frac{P(L-x)^2}{2}\\
        Q&=P(L-x)
    \end{split}
\end{equation}

则有:

\begin{equation}
    \begin{split}
        \frac{dw}{dx}+\theta_y&=\frac{kP}{GA}\left(
        L-x
        \right)\\
        \theta_y&=\frac{P\left[(L-x)^3-L^3\right]}{6EI_y}
    \end{split}
\end{equation}

\begin{equation}
    \begin{split}
        \frac{dw}{dx}&=\frac{kP}{GA}\left(
            L-x
            \right)-
            \frac{P\left[(L-x)^3-L^3\right]}{6EI_y}\\
        w&=\frac{kP}{GA}\left(
            Lx-\frac{x^2}{2}
            \right)-
            \frac{P\left[L^4-(L-x)^4-4L^3x\right]}{24EI_y}\\
        w(L)&=\frac{kPL^2}{2GA}+
            \frac{PL^4}{8EI_y}\\
    \end{split}
\end{equation}

则算例中铁木辛哥梁的端部挠度是
$\frac{1.2\times10^{-8}}{h}m^2+\frac{1.5\times10^{-8}}{h^3}m^4$


\subsection*{4.结果}

不同情况的计算结果可见表1、表2。事实上由于纯弯(见表2)时没有剪应力,
则铁木辛哥梁的精确解与欧拉-伯努利梁完全一样。在表1所示均匀载荷下,
由于有剪应力,铁木辛哥梁的精确解总比欧拉伯努利梁偏大,即更加柔软。在
$h\rightarrow 0$时两者精确解接近。



\begin{table*}[htbp]
    \footnotesize
    \begin{center}
        \caption{均布载荷下的结果}
        \label{表1}
        \begin{tabular}{|c|c|c|c|c|c|c|}
            \hline
            节点数&积分方案&单元数&$h=1m$&$h=0.1m$&$h=0.05m$&$h=0.01m$\\
            \hline
            \multirow{6}{*}{2}&\multirow{3}{*}{精确}&
            1&2.258823529411764E-08&4.715624999999999E-07&9.557057654075544E-07&4.799136207310243E-06\\
            \cline{3-7}
            &&
            10&2.693775933609955E-08&1.070823529411659E-05&4.524000000001252E-05&3.527624999999704E-04\\
            \cline{3-7}
            &&
            10000&2.699999980799952E-08&1.511999124607474E-05&1.202397282522848E-04&1.500069305525700E-02\\
            \cline{2-7}
            &\multirow{3}{*}{减缩}&
            1&2.700000000000000E-08&1.511999999999993E-05&1.202399999999978E-04&1.500120000000873E-02\\
            \cline{3-7}
            &&
            10&2.700000000000015E-08&1.511999999999777E-05&1.202399999998747E-04&1.500119999976706E-02\\
            \cline{3-7}
            &&
            10000&2.700000026594140E-08&1.511999776726433E-05&1.202399279631337E-04&1.500130257616525E-02\\
            \hline
    
            \multirow{6}{*}{3}&\multirow{3}{*}{精确}&
            1&2.661538461538462E-08&1.065571428571463E-05&8.140504854370330E-05&1.000719280872428E-02\\
            \cline{3-7}
            &&
            10&2.697448979591845E-08&1.427540540540803E-05&1.113114285715560E-04&1.375717133768484E-02\\
            \cline{3-7}
            &&
            10000&2.700000008106416E-08&1.511997036677871E-05&1.202383653075839E-04&1.500187123038968E-02\\
            \cline{2-7}
            &\multirow{3}{*}{减缩}&
            1&2.699999999999998E-08&1.512000000000094E-05&1.202400000000274E-04&1.500119999997587E-02\\
            \cline{3-7}
            &&
            10&2.699999999999939E-08&1.511999999996443E-05&1.202399999985530E-04&1.500119999628084E-02\\
            \cline{3-7}
            &&
            10000&2.699999896108366E-08&1.511998605737728E-05&1.202402612889866E-04&1.499962418362740E-02\\
            \hline
    
            \multicolumn{3}{|c|}{欧拉-伯努利解}
            &1.500000000000000E-08&1.500000000000000E-05&1.200000000000000E-04&1.500000000000000E-02\\
            \hline
            \multicolumn{3}{|c|}{铁木辛哥解}
            &2.700000000000000E-08&1.512000000000000E-05&1.202400000000000E-04&1.500120000000000E-02\\
            \hline
    
        \end{tabular}
    \end{center}
\end{table*}

\begin{table*}[htbp]
    \footnotesize
    \begin{center}
        
        \caption{一端力矩载荷下的结果}
        \label{表2}
        \begin{tabular}{|c|c|c|c|c|c|c|}
            \hline
            节点数&积分方案&单元数&$h=1m$&$h=0.1m$&$h=0.05m$&$h=0.01m$\\
            \hline
            \multirow{6}{*}{2}&\multirow{3}{*}{精确}&
            1&4.235294117647059E-08&1.406250000000000E-06&2.862823061630218E-06&1.439654482924097E-05\\
            \cline{3-7}
            &&
            10&5.975103734439814E-08&4.235294117646719E-05&1.800000000000410E-04&1.406249999999849E-03\\
            \cline{3-7}
            &&
            10000&5.999999915514277E-08&5.999996622448291E-05&4.799989509752562E-04&5.999793702477535E-02\\
            \cline{2-7}
            &\multirow{3}{*}{减缩}&
            1&6.000000000000001E-08&5.999999999999972E-05&4.799999999999911E-04&6.000000000003492E-02\\
            \cline{3-7}
            &&
            10&6.000000000000071E-08&5.999999999999142E-05&4.799999999995399E-04&5.999999999903294E-02\\
            \cline{3-7}
            &&
            10000&6.000000112649221E-08&5.999999234537814E-05&4.799997501925065E-04&6.000037607984510E-02\\
            \hline
    
            \multirow{6}{*}{3}&\multirow{3}{*}{精确}&
            1&6.000000000000003E-08&6.000000000000202E-05&4.800000000000831E-04&6.000000000056749E-02\\
            \cline{3-7}
            &&
            10&5.999999999999932E-08&6.000000000004190E-05&4.800000000000206E-04&6.000000001613761E-02\\
            \cline{3-7}
            &&
            10000&6.000000013769874E-08&5.999989495410481E-05&4.799942099727246E-04&6.000263852944890E-02\\
            \cline{2-7}
            &\multirow{3}{*}{减缩}&
            1&6.000000000000002E-08&6.000000000000267E-05&4.800000000000754E-04&5.999999999989982E-02\\
            \cline{3-7}
            &&
            10&5.999999999999899E-08&5.999999999987338E-05&4.799999999947916E-04&5.999999998625345E-02\\
            \cline{3-7}
            &&
            10000&5.999999728157785E-08&5.999994914321980E-05&4.800008377474699E-04&5.999472246346305E-02\\
            \hline
    
            \multicolumn{3}{|c|}{欧拉-伯努利解}
            &6.000000000000000E-08&6.000000000000000E-05&4.800000000000000E-04&6.000000000000000E-02\\
            \hline
    
        \end{tabular}
    \end{center}
\end{table*}

两种载荷中,都出现了剪切锁死。表1精确积分方案中当$h\rightarrow0$
发现挠度会远小于欧拉-伯努利梁的结果,即为锁死。表2即弯矩载荷中只有2节点单元出现剪切锁死,
3单元由于刚好可以表达其解析解而恰好没有锁死。此时加密网格为10个时有一定效果,
加密到10000个单元后才能趋近于精确值。

相比之下,表1所有的减缩积分都对加密不敏感。这大概是因为其欧拉-伯努利梁或者
铁木辛哥梁的精确解挠度分布比较接近线性或者二次多项式分布。(事实上欧拉-伯努利梁此时的
挠度精确解应当是一个4次多项式)

表2中,减缩积分对加密都不敏感,其可以较好表示纯弯的状态。

综合来看
均布载荷中,2节点单元的剪切锁死最为明显,而3节点的剪切锁死问题比较轻微。
力矩载荷下,3节点由于特殊情况没有明显的锁死,2节点出现明显的锁死。所有的锁死
当然
无论载荷形式,
所有的减缩积分都在1个单元时达到了较高的精度,因此精确积分此时是基本不可取的。

综合来看,采用3节点单元相比2节点可以取得更好的精度,在较小的单元数下可以表达更复杂
的变形情况。减缩积分可以消除剪切锁死的问题,采用减缩积分的算例,对于以上的简单载荷,
即使是在1单元时也可以达到较好的精度。此外,如果不考虑计算成本,剪切锁死也可以通过加密
网格来缓解,但以上结果表明其效果不明显,代价很大。

此外,表2中很多结果与欧拉-伯努利梁有接近机器精度的误差而表1不是如此,这是因为表1中
对应的精确解本身就和欧拉-伯努利解有差距,而表2的精确解恰好和欧拉-伯努利梁相同。表1中
改用铁木辛哥解进行对比,则减缩积分在单元不太多时基本可以达到机器精度。

\subsection*{5.总结}

本文讨论了采用减缩积分方法的铁木辛哥梁直梁的有限元方法,探讨了积分方案、单元数目、节点
数目、梁的高度和载荷形式对解的影响。实验表明精确积分会导致剪切锁死,而剪切积分基本上规避
了这个问题。同时,实验还表明了剪切锁死可以通过加密网格缓解,但效率不佳。

\subsection*{6.代码说明}

提交代码为MATALB脚本。

运行run.m脚本可以计算某个算例,其中调用的CASE*代表的是算例配置。

调用的CASE*.m文件中可以定义更加详细的算例配置。

采用记号‘nn[p/r]nelem‘来表示节点、积分和单元数(均匀划分)的
方案,例如:‘2r1’是2节点单元,减缩积分,1个单元;‘3p10’是3节点单元,精确
积分,10个单元。

记号‘h[height]‘代表算例选取的梁高度,如‘h0.1’代表梁高0.1m。

记号‘L1’代表载荷为$1N/m$的均布力,‘L2’代表载荷为$1N\cdot m$的端部
弯矩。

代码为单进程,建议自由度不超过5000000。

CantileverTimo.m是单独的计算过程,包括了静力问题和动力学模态问题。
其中绘制了挠度曲线和模态曲线。经过
验证,在减缩积分方案中,20单元下前5阶模态频率计算与铁木辛哥梁理论解吻合。


% \begin{equation}
%     \begin{split}
%         A^H&=A\\
%         AV&=VD\\
%         v^HAv&=v^Hv\lambda\\
%         \overline{v^HAv}&=v^TA^Tv^HT=(v^HAv)^T=v^HAv=\text{somereal}\\
%         v^Hv&=\text{somereal}\\
%         \Rightarrow \lambda&=\text{somereal}\\
%         u^HAv&=u^Hv\lambda_v=v^HAu=v^Hu\lambda_u=\overline{u^Hv}\lambda_u\\
%         \Rightarrow u^Hv&=0
%     \end{split}
% \end{equation}


\end{document}