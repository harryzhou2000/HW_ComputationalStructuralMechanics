%!TEX program = xelatex
\documentclass[UTF8,c5size]{ctexart}


\title{{\bfseries 第二次作业}}
\author{周涵宇 2018011600}
\date{}

\usepackage[a4paper]{geometry}
\geometry{left=0.75in,right=0.75in,top=1in,bottom=1in}

\usepackage[
UseMSWordMultipleLineSpacing,
MSWordLineSpacingMultiple=1.5
]{zhlineskip}

\usepackage{fontspec}
\setmainfont{Cambria Math}
% \setmonofont{JetBrains Mono}
\setCJKmainfont{仿宋}[AutoFakeBold=true]
\setCJKsansfont{黑体}[AutoFakeBold=true]

\usepackage{bm}
\usepackage{amsmath}

\begin{document}

\maketitle

考虑欧拉-伯努利梁直梁单元,$x$ 为梁的方向,2节点单元,扭转
转角、拉伸位移采用拉格朗日插值,弯曲位移与弯曲转角共同通过
Hermite三次多项式插值及其导数取得,单元自由度记为:

\begin{equation}
\bm{a_e}=[u^1\ \ v^1\ \ w^1\ \ \theta^1_x\ \ \theta^1_y\ \ \theta^1_z
\ \ u^2\ \ v^2\ \ w^2\ \ \theta^2_x\ \ \theta^2_y\ \ \theta^2_z]^T
\end{equation}

拉格朗日2节点一维插值为:

\begin{equation}
N^1(\xi) = 1+\xi,\ \ N^2(\xi) = 1-\xi
\end{equation}

Hermite插值为:

\begin{equation}
\begin{split}
N^1(\xi)&=\ \ \ \frac{1}{2}-\frac{3}{4}\xi+\frac{1}{4}\xi^3\\
N^2(\xi)&=\ \ \ \frac{1}{4}-\frac{1}{4}\xi-\frac{1}{4}\xi^2+\frac{1}{4}\xi^3\\
N^3(\xi)&=\ \ \ \frac{1}{2}+\frac{3}{4}\xi-\frac{1}{4}\xi^3\\
N^4(\xi)&=-\frac{1}{4}-\frac{1}{4}\xi+\frac{1}{4}\xi^2+\frac{1}{4}\xi^3
\end{split}
\end{equation}

其中,$1,2$对应$\xi=-1$的函数值和一阶导,$3,4$对应$\xi=1$。

欧拉-伯努利直梁内的总势能为:

\begin{equation}
    \begin{split}
        \Pi_p=
            &\int_{l}{\left[
            \frac{1}{2}EI_y\left(\frac{d^2w}{dx^2}\right)^2+
            \frac{1}{2}EI_z\left(\frac{d^2v}{dx^2}\right)^2+
            \frac{1}{2}EA\left(\frac{du}{dx}\right)^2+
            \frac{1}{2}GI_x\left(\frac{d\theta_x}{dx}\right)^2+            
            \right]dl}+\\
            &+\int_{l}{\left[
                uP_x+vP_y+wP_z+\theta_xMx+\theta_yMy+\theta_zMz
                \right]dl}\\
            &+\sum_{\forall i_c}{\left[
            u_{i_c}P_{xi_c}+v_{i_c}P_{vi_c}+w_{i_c}P_{zi_c}
            +\theta_{xi_c}M_{xi_c}+\theta_{yi_c}M_{yi_c}+\theta_{zi_c}M_{zi_c}
            \right]}
    \end{split}
\end{equation}

势能变分原理为

$$
\delta\Pi_p=0
$$

其中要求位移满足可能,即边界满足位移约束,转角满足:

\begin{equation}
    \theta_y=-\frac{dw}{dx},\ \ \ \ \theta_z=\frac{dv}{dx}
\end{equation}





    
\end{document}