%!TEX program = xelatex
\documentclass[UTF8,c5size]{ctexart}


\title{{\bfseries 第二次作业}}
\author{周涵宇 2018011600}
\date{}

\usepackage[a4paper]{geometry}
\geometry{left=0.75in,right=0.75in,top=1in,bottom=1in}

\usepackage[
UseMSWordMultipleLineSpacing,
MSWordLineSpacingMultiple=1.5
]{zhlineskip}

\usepackage{fontspec}
\setmainfont{Cambria Math}
% \setmonofont{JetBrains Mono}
\setCJKmainfont{仿宋}[AutoFakeBold=true]
\setCJKsansfont{黑体}[AutoFakeBold=true]

\usepackage{bm}
\usepackage{amsmath}
\usepackage{array}

\begin{document}

\maketitle

考虑欧拉-伯努利梁直梁单元,$x$ 为梁的方向,2节点单元,扭转
转角、拉伸位移采用拉格朗日插值,弯曲位移与弯曲转角共同通过
Hermite三次多项式插值及其导数取得,单元自由度记为:

\begin{equation}
\bm{a_e}=[u^1\ \ v^1\ \ w^1\ \ \theta^1_x\ \ \theta^1_y\ \ \theta^1_z
\ \ u^2\ \ v^2\ \ w^2\ \ \theta^2_x\ \ \theta^2_y\ \ \theta^2_z]^T
\end{equation}

拉格朗日2节点一维插值为:

\begin{equation}
N^1(\xi) = \frac{1-\xi}{2},\ \ N^2(\xi) = \frac{1+\xi}{2}
\end{equation}

Hermite插值为:

\begin{equation}
\begin{split}
N^1_H(\xi)&=\ \ \ \frac{1}{2}-\frac{3}{4}\xi+\frac{1}{4}\xi^3\\
N^2_H(\xi)&=\ \ \ \frac{1}{4}-\frac{1}{4}\xi-\frac{1}{4}\xi^2+\frac{1}{4}\xi^3\\
N^3_H(\xi)&=\ \ \ \frac{1}{2}+\frac{3}{4}\xi-\frac{1}{4}\xi^3\\
N^4_H(\xi)&=-\frac{1}{4}-\frac{1}{4}\xi+\frac{1}{4}\xi^2+\frac{1}{4}\xi^3
\end{split}
\end{equation}

其中,$1,2$对应$\xi=-1$的函数值和{\bfseries 参数空间}内的一阶导,$3,4$对应$\xi=1$。

以上插值都定义在$[-1,1]$。

欧拉-伯努利直梁内的总势能为:

\begin{equation}
    \begin{split}
        \Pi_p=
            &\int_{l}{\left[
            \frac{1}{2}EI_y\left(\frac{d^2w}{dx^2}\right)^2+
            \frac{1}{2}EI_z\left(\frac{d^2v}{dx^2}\right)^2+
            \frac{1}{2}EA\left(\frac{du}{dx}\right)^2+
            \frac{1}{2}GI_x\left(\frac{d\theta_x}{dx}\right)^2+            
            \right]dl}+\\
            &+\int_{l}{\left[
                uP_x+vP_y+wP_z+\theta_xMx+\theta_yMy+\theta_zMz
                \right]dl}\\
            &+\sum_{\forall i_c}{\left[
            u_{i_c}P_{xi_c}+v_{i_c}P_{vi_c}+w_{i_c}P_{zi_c}
            +\theta_{xi_c}M_{xi_c}+\theta_{yi_c}M_{yi_c}+\theta_{zi_c}M_{zi_c}
            \right]}
    \end{split}
\end{equation}

势能变分原理为

$$
\delta\Pi_p=0
$$

其中要求位移满足可能,即边界满足位移约束,转角满足:

\begin{equation}
    \theta_y=-\frac{dw}{dx},\ \ \ \ \theta_z=\frac{dv}{dx}
\end{equation}

通过以上泛函变分原理构建有限元格式,与刚度阵有关部分是其二次部分,
有单元刚度阵如:

\begin{equation}
    \begin{split}
    \bm{K_e} = &\int_{l}{\left[
        EI_z\left(\frac{d^2\bm{N_y}}{dx^2}\right)^T
        \left(\frac{d^2\bm{N_y}}{dx^2}\right)+
        EI_y\left(\frac{d^2\bm{N_z}}{dx^2}\right)^T
        \left(\frac{d^2\bm{N_z}}{dx^2}\right)
        \right]dl}\\+
    &\int_{l}{\left[
        EA\left(\frac{d\bm{N_{u}}}{dx}\right)^T
        \left(\frac{d\bm{N_{u}}}{dx}\right)+
        GI_x\left(\frac{d\bm{N_{\theta_x}}}{dx}\right)^T
        \left(\frac{d\bm{N_{\theta_x}}}{dx}\right)
    \right]dl}
    \end{split}
\end{equation}

对于2节点单元参数空间-物理空间采用线性变换,则可知有:


$$
\int_{L}{fdl}=\frac{l}{2}\int_{-1}^{1}{fd\xi}
$$


以及:

$$
\frac{df}{dx} = \frac{2}{l}\bullet\frac{df}{d\xi}
$$

则可知其中,形函数矩阵分别定义为:

\begin{equation}
    \bm{N_y}(\xi) = 
    \addtocounter{MaxMatrixCols}{40}
    \begin{bmatrix}
        0&N^1_H&0&0&0&\frac{l}{2}N^2_H&0&N^3_H&0&0&0&\frac{l}{2}N^4_H
    \end{bmatrix}
\end{equation}

\begin{equation}
    \bm{N_z}(\xi) = 
    \addtocounter{MaxMatrixCols}{40}
    \begin{bmatrix}
        0&0&N^1_H&0&-\frac{l}{2}N^2_H&0&0&0&N^3_H&0&-\frac{l}{2}N^4_H&0
    \end{bmatrix}
\end{equation}

\begin{equation}
    \bm{N_{u}}(\xi) = 
    \addtocounter{MaxMatrixCols}{40}
    \begin{bmatrix}
        N^1&0&0&0&0&0&N^2&0&0&0&0&0
    \end{bmatrix}
\end{equation}

\begin{equation}
    \bm{N_{\theta_x}}(\xi) = 
    \addtocounter{MaxMatrixCols}{40}
    \begin{bmatrix}
        0&0&0&N^1&0&0&0&0&0&N^2&0&0
    \end{bmatrix}
\end{equation}

注意以上由于所有的形函数矩阵需要右乘物理自由度,而物理自由度中的一阶导并不是参数空间的一阶导,
其变换需要事先确定参数变换(比如等参变换等方案),此处是线性的所以直接通过缩放来获得。
同时注意以上角度和一阶导的对应关系,包括(8)中的符号问题。

此时,假设梁的弹性常数和截面都是均匀的,则积分可得:

\begin{equation}
    \begin{split}
        &\left[\begin{array}{cccccccccccc} \frac{A\,E}{l} & 0 & 0 & 0 & 0 & 0 & -\frac{A\,E}{l} & 0 & 0 & 0 & 0 & 0\\ 0 & \frac{12\,E\,I_{z}}{l^3} & 0 & 0 & 0 & \frac{6\,E\,I_{z}}{l^2} & 0 & -\frac{12\,E\,I_{z}}{l^3} & 0 & 0 & 0 & \frac{6\,E\,I_{z}}{l^2}\\ 0 & 0 & \frac{12\,E\,I_{y}}{l^3} & 0 & -\frac{6\,E\,I_{y}}{l^2} & 0 & 0 & 0 & -\frac{12\,E\,I_{y}}{l^3} & 0 & -\frac{6\,E\,I_{y}}{l^2} & 0\\ 0 & 0 & 0 & \frac{G\,I_{x}}{l} & 0 & 0 & 0 & 0 & 0 & -\frac{G\,I_{x}}{l} & 0 & 0\\ 0 & 0 & -\frac{6\,E\,I_{y}}{l^2} & 0 & \frac{4\,E\,I_{y}}{l} & 0 & 0 & 0 & \frac{6\,E\,I_{y}}{l^2} & 0 & \frac{2\,E\,I_{y}}{l} & 0\\ 0 & \frac{6\,E\,I_{z}}{l^2} & 0 & 0 & 0 & \frac{4\,E\,I_{z}}{l} & 0 & -\frac{6\,E\,I_{z}}{l^2} & 0 & 0 & 0 & \frac{2\,E\,I_{z}}{l}\\ -\frac{A\,E}{l} & 0 & 0 & 0 & 0 & 0 & \frac{A\,E}{l} & 0 & 0 & 0 & 0 & 0\\ 0 & -\frac{12\,E\,I_{z}}{l^3} & 0 & 0 & 0 & -\frac{6\,E\,I_{z}}{l^2} & 0 & \frac{12\,E\,I_{z}}{l^3} & 0 & 0 & 0 & -\frac{6\,E\,I_{z}}{l^2}\\ 0 & 0 & -\frac{12\,E\,I_{y}}{l^3} & 0 & \frac{6\,E\,I_{y}}{l^2} & 0 & 0 & 0 & \frac{12\,E\,I_{y}}{l^3} & 0 & \frac{6\,E\,I_{y}}{l^2} & 0\\ 0 & 0 & 0 & -\frac{G\,I_{x}}{l} & 0 & 0 & 0 & 0 & 0 & \frac{G\,I_{x}}{l} & 0 & 0\\ 0 & 0 & -\frac{6\,E\,I_{y}}{l^2} & 0 & \frac{2\,E\,I_{y}}{l} & 0 & 0 & 0 & \frac{6\,E\,I_{y}}{l^2} & 0 & \frac{4\,E\,I_{y}}{l} & 0\\ 0 & \frac{6\,E\,I_{z}}{l^2} & 0 & 0 & 0 & \frac{2\,E\,I_{z}}{l} & 0 & -\frac{6\,E\,I_{z}}{l^2} & 0 & 0 & 0 & \frac{4\,E\,I_{z}}{l} \end{array}\right]
        \\
        &=\bm{K_e}
    \end{split}
\end{equation}

给出了单元刚度阵。

事实上,小变形直梁里面单元刚度阵是完全解耦的,拉伸、弯曲和扭转共4部分。
梁的几何是弯曲时坐标变换使其各部分出现耦合。
    
\end{document}